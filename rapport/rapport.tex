\documentclass[a4paper, 10pt, french]{article}
% Préambule; packages qui peuvent être utiles
\RequirePackage[T1]{fontenc}        % Ce package pourrit les pdf...
\RequirePackage{babel,indentfirst}  % Pour les césures correctes,
% et pour indenter au début de chaque paragraphe
\RequirePackage[utf8]{inputenc}   % Pour pouvoir utiliser directement les accents
% et autres caractères français
\RequirePackage{lmodern,tgpagella} % Police de caractères
\textwidth 17cm \textheight 25cm \oddsidemargin -0.24cm % Définition taille de la page
\evensidemargin -1.24cm \topskip 0cm \headheight -1.5cm % Définition des marges

\RequirePackage{latexsym}               % Symboles
\RequirePackage{amsmath}                   % Symboles mathématiques
\RequirePackage{tikz}   % Pour faire des schémas
\RequirePackage{graphicx} % Pour inclure des images
\RequirePackage{listings} % pour mettre des listings
% Fin Préambule; package qui peuvent être utiles

\title {Rapport de TP APMA : Jeu à deux joueurs à somme nulle et à information complète}
\author
{
    MAHIEU LUCAS
    \\ DE VALON HUGUES
}

\begin{document}

\maketitle

%%%%%%%%%%%%%%%%%%%%%%%%%%%%%%%%%%%%%%%%%%%%%%
\section{Implémentation des structures de données}
{
    % Présentation
    % Tableau de joueurs
    % Liste Générique
    % Généricité

    Notre programme applique des packages de jeux génériques au jeu du puissance 4.
    Nous allons ci-dessous décrire les structures de données principales utilisées, les algorithmes importants et nos résultats.

    \subsection{Tableau de joueurs}
    {
        Pour faciliter la représentation du tableau et de tous les calculs dans notre code, nous avons ajouté à l'énumération
        Joueur un dernier membre Vide. Ainsi, pour représenter le tableau (structure Etat), nous avons utilisé un tableau 
        de Joueur de taille $Largeur  \times Hauteur$ et chaque case peut donc être : Joueur1, Joueur2 ou Vide. 
        Ceci a grandement facilité tout nos algorithmes d'affichage, et de calcul d'état. De plus, les coups étant joué à la suite
        et à cause de la gravité terrestre, nous avons choisi de représenter Coup par un Joueur et une colonne du tableau. Il est
        alors simple de savoir où est le pion que le joueur vient de poser.
    }

    \subsection{Listes}
    {
        Le package Liste Générique que nous avons créé permet de créer ou de supprimer une liste, d'y ajouter des éléments,
        ainsi que de de gérer des itérateurs. L'utilisation des itérateurs est particulièrement utile pour s'affranchir 
        du mécanisme des listes. Un fois créé, il suffit de lui appliquer la fonction {\em Suivant} pour le placer sur 
        l'élément suivant ou {\em Element\_Courant} pour récupérer la valeur courante de la liste. 
        Cela permet une utilisation plus simple et plus sûre des listes.
    }

    \subsection{Utilisation des packages}
    {
        Afin de généraliser notre code à plusieurs jeux, nous utilisons avec force la généricité dans les packages crées.
        En effet, nous utilisons dans chaque fichier .adb, seulement les fonctions et procédures déclarées dans les fichiers
        .ads, qu'elles soient génériques ou non. Cette façon de faire nous permet que tout fonctionne correctement en ne changeant
        uniquement que le package Puissance4 (et le remplacer par un autre jeu éventuellement).
        C'est pourquoi, dans la procédure mère main, on instancie d'abord le package Puissance4 en précisant ses paramètres
        génériques (largeur, hauteur et nombre de pions nécessaires) puis on utilise les fonctions de Puissance4 dans Partie.
        Si on veut utiliser l'ordinateur pour remplacer des joueurs, on aura juste à créer le package Moteur avec les fonctions
        de Puissance4 et utiliser pour Partie la fonction dans Moteur qui renvoit le coup à jouer pour l'ordinateur à la place de
        celle qui le demandait à l'utilisateur.
        Dans notre code, seules le package Puissance4 est unique à ce jeu et tout le reste peut être gardé si on change de jeu.
    }
} 

%%%%%%%%%%%%%%%%%%%%%%%%%%%%%%%%%%%%%%%%%%%%%%
\section{Algorithmes important}
{
    % Est_Gagnant
    % Eval
    % Min_Max
    % Aléatoire
    % Partie

    \subsection{Déroulement d'une partie}
    {
        C'est la procédure Joue\_Partie qui lance la partie en commençant par le Joueur donné en paramètre. On effectue une boucle
        while qui vérifie que l'état n'est pas en status quo puis on passe d'un joueur à l'autre en effectuant à chaque fois
        la même séquence de fonctions du package Partie :
        \begin{enumerate}
            \item Affichage du jeu
            \item Demande du coup
            \item Affichage du coup
            \item Calcul de l'état suivant étant donné le coup
            \item Vérification si l'état est gagnant
            \item Tour suivant : rappel de la fonction pour l'autre joueur.
        \end{enumerate}
        Grâce à la généricité, la fonction demande le coup suivant au joueur, ou alors c'est l'ordinateur qui le calcule.
        Si l'état est gagnant, notre code effectue un goto pour sortir de la boucle principale et quitter le programme.
        A l'avantage d'un exit, cette méthode permet d'afficher directement dans la boucle le message de victoire au lieu
        d'avoir à appeler encore une fois la fonction de vérification de l'état gagnant ou d'utiliser des méthodes moins
        compréhensibles que le très controversé goto.
    }

    \subsection{Vérification de l'état gagnant}
    {
        Comme la fonction Est\_Gagnant est appelée après chaque coup, il est inutile de vérifier un alignement de pion pour
        tous les pions de la grille. On ne vérifie alors l'alignement seulement pour les pions que le Joueur donné en paramètre
        est susceptible d'avoir posé, c'est à dire les pions du joueur aux sommets des piles de pions : les dernières cases
        non vides.
        Pour ces pions là, on vérifie ses voisins pour trouver un alignement de pions suffisant pour la victoire en effectuant
        des boucles while (E(i\_test, j\_test) = J) et en incrémentant un comtpeur à chaque itération. On renvoit true dès que
        on a trouvé un tel alignement. On vérifie les directions suivantes à chaque fois :
        \begin{enumerate}
            \item Diagonale $y = x$ (sens montant et descendant)
            \item Diagonale $y = -x$ (sens montant et descendant)
            \item Ligne horizontale (à droite et à gauche)
            \item Colonne (sous le pion posé)
        \end{enumerate}
        On vérifie bien sur à chaque fois que l'on ne dépasse pas les dimensions du tableau en faisant des exit sur ces boucles
        while quand nécessaires. Si on n'a pas trouvé d'alignements, on renvoit false.
        On peut borner le coût en pire cas de cet algorithme par :
        \begin{center}
            $7 \times Largeur \times Pions\_Victoires$
        \end{center}
        7 correspond à tous les sens exploré pour chaque pion.
    }

    \subsection{Evaluation statique}
    {
        L'évaluation statique est le coeur de l'algorithmique Min-Max et donc l'efficacité de cette fonction correspond à l'
        efficacité du jeu de l'ordinateur. Notre fonction renvoit Pions\_Victoire en cas d'état gagnant pour le Joueur ou
        l'opposé pour l'adversaire.
        Dans les autres cas : on parcourt tout le tableau et on regarde pour chaque pion s'il fait partie d'une séquence
        complétable. On entend par séquence complétable un alignement de pions qui peut être complété par un nouveau pion.
        Pour cela, il faut bien vérifier que le pion que l'on souhaite ajouter ne repose pas sur une case Vide : il faut un pion au
        dessous pour le supporter.
        De même que pour la fonction Est\_Gagnant, on regarde dans toutes les directions et prenant aussi la colonne montante (sur
        le pion considéré) et on garde la taille de la plus grande séquence complétable que l'on trouve.
        A la fin du parcours de tout le tableau, on a pour les deux joueurs, la taille de leur plus grande séquence complétable
        on renvoit 0 si ces tailles sont identiques et + ou - la taille suivant que c'est le JoueurMoteur qui a la plus grande
        ou l'adversaire.
        Cette méthode fournit une bonne évaluation statique du jeu, cependant, on ne traite pas les cas où la séquence n'est pas
        complétable quand il n'y a pas assez de place et on ne compte pas le nombre de séquence de même taille qu'on rencontre.
        En effet si le Joueur1 a 3 séquence de 3 pions alignés, l'évaluation lui devrait être favorable par rapport au Joueur2
        qui aurait qu'une seule séquence de 3 pions. Cette distinction n'est pas faite car en parcourant tous les pions on
        compterait une même séquence plusieures fois.
        Le coût de cet algorithme en pire cas peut être borné par :
        \begin{center}
            $14 \times Largeur \times Pions\_Victoires + 8 \times Largeur \times Hauteur \times (Pions\_Victoire - 1)$
        \end{center}
        Cette fonction vérifiant si le Joueur ou l'adversaire ayant gagné, on doit refaire les mêmes opérations deux fois.
        Ensuite, on vérifie pour les 4 directions et dans les deux sens tous les pions de la grille avec des while qui
        s'executent au maximum Pions\_Victoire - 1.
    }

    \subsection{Algorithme Min-Max}
    {
    }

    \subsection{Génération de l'aléatoire}
    {
    }
}   

\section{Résultats}
{
    % Résultats généraux
    % Ordinateur contre lui-même

    \subsection{Résultats généraux}
    {
    }

    \subsection{Influence de la profondeur de recherche}
    {
    }
}

\end{document}
%% Fin mise au format
