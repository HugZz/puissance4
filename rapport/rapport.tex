\documentclass[a4paper, 10pt, french]{article}
% Préambule; packages qui peuvent être utiles
\RequirePackage[T1]{fontenc}        % Ce package pourrit les pdf...
\RequirePackage{babel,indentfirst}  % Pour les césures correctes,
% et pour indenter au début de chaque paragraphe
\RequirePackage[utf8]{inputenc}   % Pour pouvoir utiliser directement les accents
% et autres caractères français
\RequirePackage{lmodern,tgpagella} % Police de caractères
\textwidth 17cm \textheight 25cm \oddsidemargin -0.24cm % Définition taille de la page
\evensidemargin -1.24cm \topskip 0cm \headheight -1.5cm % Définition des marges

\RequirePackage{latexsym}               % Symboles
\RequirePackage{amsmath}                   % Symboles mathématiques
\RequirePackage{tikz}   % Pour faire des schémas
\RequirePackage{graphicx} % Pour inclure des images
\RequirePackage{listings} % pour mettre des listings
% Fin Préambule; package qui peuvent être utiles

\title {Rapport de TP APMA : Jeu à deux joueurs à somme nulle et à information complète}
\author
{
    MAHIEU LUCAS
    \\ DE VALON HUGUES
}

\begin{document}

\maketitle

%%%%%%%%%%%%%%%%%%%%%%%%%%%%%%%%%%%%%%%%%%%%%%
\section{Implémentation des structures de données}
{
    % Présentation
    % Tableau de joueurs
    % Liste Générique
    Notre programme applique des packages de jeux génériques au jeu du puissance 4.
    Nous allons ci-dessous décrire chacun de ces packages.
    Le Package Liste Générique que nous avons créé permet de créer ou supprimer une liste, d'y ajouter des éléments, ainsi que de de gérer des itérateurs. L'utilisation des Itérateurs est particulièrement utiles pour s'affranchir du mécanisme des listes. Un fois créé, il suffie de lui appliquer la fonction {\em Suivant} pour le placer sur l'élément suivant ou {\em Element\_Courant} pour récupérer la valeur courante de la liste. Cela permet une utilisation plus simple et plus sûre de liste.
} 

%%%%%%%%%%%%%%%%%%%%%%%%%%%%%%%%%%%%%%%%%%%%%%
\section{Algorithmes important}
{
    % Est_Gagnant
    % Eval
    % Min_Max
    % Aléatoire
    % Partie
    \subsection{Place mémoire requise\,: }
    {

    }
}   

\section{Résultats}
{
    % Résultats généraux
    % Ordinateur contre lui-même
}

\end{document}
%% Fin mise au format
